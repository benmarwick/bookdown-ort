\documentclass[]{book}
\usepackage{lmodern}
\usepackage{amssymb,amsmath}
\usepackage{ifxetex,ifluatex}
\usepackage{fixltx2e} % provides \textsubscript
\ifnum 0\ifxetex 1\fi\ifluatex 1\fi=0 % if pdftex
  \usepackage[T1]{fontenc}
  \usepackage[utf8]{inputenc}
\else % if luatex or xelatex
  \ifxetex
    \usepackage{mathspec}
  \else
    \usepackage{fontspec}
  \fi
  \defaultfontfeatures{Ligatures=TeX,Scale=MatchLowercase}
\fi
% use upquote if available, for straight quotes in verbatim environments
\IfFileExists{upquote.sty}{\usepackage{upquote}}{}
% use microtype if available
\IfFileExists{microtype.sty}{%
\usepackage{microtype}
\UseMicrotypeSet[protrusion]{basicmath} % disable protrusion for tt fonts
}{}
\usepackage[margin=1in]{geometry}
\usepackage{hyperref}
\hypersetup{unicode=true,
            pdftitle={A Minimal Book Example},
            pdfauthor={Yihui Xie},
            pdfborder={0 0 0},
            breaklinks=true}
\urlstyle{same}  % don't use monospace font for urls
\usepackage{natbib}
\bibliographystyle{apalike}
\usepackage{color}
\usepackage{fancyvrb}
\newcommand{\VerbBar}{|}
\newcommand{\VERB}{\Verb[commandchars=\\\{\}]}
\DefineVerbatimEnvironment{Highlighting}{Verbatim}{commandchars=\\\{\}}
% Add ',fontsize=\small' for more characters per line
\usepackage{framed}
\definecolor{shadecolor}{RGB}{248,248,248}
\newenvironment{Shaded}{\begin{snugshade}}{\end{snugshade}}
\newcommand{\KeywordTok}[1]{\textcolor[rgb]{0.13,0.29,0.53}{\textbf{{#1}}}}
\newcommand{\DataTypeTok}[1]{\textcolor[rgb]{0.13,0.29,0.53}{{#1}}}
\newcommand{\DecValTok}[1]{\textcolor[rgb]{0.00,0.00,0.81}{{#1}}}
\newcommand{\BaseNTok}[1]{\textcolor[rgb]{0.00,0.00,0.81}{{#1}}}
\newcommand{\FloatTok}[1]{\textcolor[rgb]{0.00,0.00,0.81}{{#1}}}
\newcommand{\ConstantTok}[1]{\textcolor[rgb]{0.00,0.00,0.00}{{#1}}}
\newcommand{\CharTok}[1]{\textcolor[rgb]{0.31,0.60,0.02}{{#1}}}
\newcommand{\SpecialCharTok}[1]{\textcolor[rgb]{0.00,0.00,0.00}{{#1}}}
\newcommand{\StringTok}[1]{\textcolor[rgb]{0.31,0.60,0.02}{{#1}}}
\newcommand{\VerbatimStringTok}[1]{\textcolor[rgb]{0.31,0.60,0.02}{{#1}}}
\newcommand{\SpecialStringTok}[1]{\textcolor[rgb]{0.31,0.60,0.02}{{#1}}}
\newcommand{\ImportTok}[1]{{#1}}
\newcommand{\CommentTok}[1]{\textcolor[rgb]{0.56,0.35,0.01}{\textit{{#1}}}}
\newcommand{\DocumentationTok}[1]{\textcolor[rgb]{0.56,0.35,0.01}{\textbf{\textit{{#1}}}}}
\newcommand{\AnnotationTok}[1]{\textcolor[rgb]{0.56,0.35,0.01}{\textbf{\textit{{#1}}}}}
\newcommand{\CommentVarTok}[1]{\textcolor[rgb]{0.56,0.35,0.01}{\textbf{\textit{{#1}}}}}
\newcommand{\OtherTok}[1]{\textcolor[rgb]{0.56,0.35,0.01}{{#1}}}
\newcommand{\FunctionTok}[1]{\textcolor[rgb]{0.00,0.00,0.00}{{#1}}}
\newcommand{\VariableTok}[1]{\textcolor[rgb]{0.00,0.00,0.00}{{#1}}}
\newcommand{\ControlFlowTok}[1]{\textcolor[rgb]{0.13,0.29,0.53}{\textbf{{#1}}}}
\newcommand{\OperatorTok}[1]{\textcolor[rgb]{0.81,0.36,0.00}{\textbf{{#1}}}}
\newcommand{\BuiltInTok}[1]{{#1}}
\newcommand{\ExtensionTok}[1]{{#1}}
\newcommand{\PreprocessorTok}[1]{\textcolor[rgb]{0.56,0.35,0.01}{\textit{{#1}}}}
\newcommand{\AttributeTok}[1]{\textcolor[rgb]{0.77,0.63,0.00}{{#1}}}
\newcommand{\RegionMarkerTok}[1]{{#1}}
\newcommand{\InformationTok}[1]{\textcolor[rgb]{0.56,0.35,0.01}{\textbf{\textit{{#1}}}}}
\newcommand{\WarningTok}[1]{\textcolor[rgb]{0.56,0.35,0.01}{\textbf{\textit{{#1}}}}}
\newcommand{\AlertTok}[1]{\textcolor[rgb]{0.94,0.16,0.16}{{#1}}}
\newcommand{\ErrorTok}[1]{\textcolor[rgb]{0.64,0.00,0.00}{\textbf{{#1}}}}
\newcommand{\NormalTok}[1]{{#1}}
\usepackage{longtable,booktabs}
\usepackage{graphicx,grffile}
\makeatletter
\def\maxwidth{\ifdim\Gin@nat@width>\linewidth\linewidth\else\Gin@nat@width\fi}
\def\maxheight{\ifdim\Gin@nat@height>\textheight\textheight\else\Gin@nat@height\fi}
\makeatother
% Scale images if necessary, so that they will not overflow the page
% margins by default, and it is still possible to overwrite the defaults
% using explicit options in \includegraphics[width, height, ...]{}
\setkeys{Gin}{width=\maxwidth,height=\maxheight,keepaspectratio}
\IfFileExists{parskip.sty}{%
\usepackage{parskip}
}{% else
\setlength{\parindent}{0pt}
\setlength{\parskip}{6pt plus 2pt minus 1pt}
}
\setlength{\emergencystretch}{3em}  % prevent overfull lines
\providecommand{\tightlist}{%
  \setlength{\itemsep}{0pt}\setlength{\parskip}{0pt}}
\setcounter{secnumdepth}{5}
% Redefines (sub)paragraphs to behave more like sections
\ifx\paragraph\undefined\else
\let\oldparagraph\paragraph
\renewcommand{\paragraph}[1]{\oldparagraph{#1}\mbox{}}
\fi
\ifx\subparagraph\undefined\else
\let\oldsubparagraph\subparagraph
\renewcommand{\subparagraph}[1]{\oldsubparagraph{#1}\mbox{}}
\fi

%%% Use protect on footnotes to avoid problems with footnotes in titles
\let\rmarkdownfootnote\footnote%
\def\footnote{\protect\rmarkdownfootnote}

%%% Change title format to be more compact
\usepackage{titling}

% Create subtitle command for use in maketitle
\newcommand{\subtitle}[1]{
  \posttitle{
    \begin{center}\large#1\end{center}
    }
}

\setlength{\droptitle}{-2em}
  \title{A Minimal Book Example}
  \pretitle{\vspace{\droptitle}\centering\huge}
  \posttitle{\par}
  \author{Yihui Xie}
  \preauthor{\centering\large\emph}
  \postauthor{\par}
  \predate{\centering\large\emph}
  \postdate{\par}
  \date{2016-12-12}

\usepackage{booktabs}

\let\BeginKnitrBlock\begin \let\EndKnitrBlock\end
\begin{document}
\maketitle

{
\setcounter{tocdepth}{1}
\tableofcontents
}
\chapter{Prerequisites}\label{prerequisites}

This is a \emph{sample} book written in \textbf{Markdown}. You can use
anything that Pandoc's Markdown supports, e.g., a math equation
\(a^2 + b^2 = c^2\).

For now, you have to install the development versions of
\textbf{bookdown} from Github:

\begin{Shaded}
\begin{Highlighting}[]
\NormalTok{devtools::}\KeywordTok{install_github}\NormalTok{(}\StringTok{"rstudio/bookdown"}\NormalTok{)}
\end{Highlighting}
\end{Shaded}

Remember each Rmd file contains one and only one chapter, and a chapter
is defined by the first-level heading \texttt{\#}.

To compile this example to PDF, you need to install XeLaTeX.

\section{Modifications}\label{modifications}

In this template book, I have made the following modifications inspired
by \href{https://twitter.com/msalganik}{Matthew Salganik's}
\href{http://www.openreviewtoolkit.org/}{Open Review Toolkit}. The ORT
is an excellent project to enable low-friction engagement between the
author and readers. However, it pre-dates bookdown and lacks some of its
flexibility, so this is an effort to bring some of the features of the
ORT into bookdown. I've made three modifications: an open review block,
a call-to-action block, and added google analytics.

\subsection{Open review block}\label{open-review-block}

This is a custom block that can appear at the top of each page informing
the reader that they can give immediate feedback directly on the page.
It looks like this:

\BeginKnitrBlock{rmdreview}
This book is in Open Review. I want your feedback to make the book
better for you and other readers. To add your annotation, {select some
text} and then click the on the pop-up menu. To see the annotations of
others, click the in the upper right hand corner of the page
\EndKnitrBlock{rmdreview}

The feedback tool is \href{https://hypothes.is/}{hypothes.is}, which
enables sentence-level annotation directly on the page. There is a
script in \texttt{\_includes/hypothesis.html}, which we call in
\texttt{\_output.yml} like this:

\begin{verbatim}
bookdown::gitbook:
  includes:
    in_header: [_includes/hypothesis.html, _includes/google_analytics.html]
  css: [_includes/style.css, _includes/font-awesome.css]
\end{verbatim}

The \texttt{hypothesis.html} file contains a short script that activates
the tool on this site. The ORT also had code to track the use of
\href{https://hypothes.is/}{hypothes.is}, but I don't understand how
that works, so I haven't done it here. To make the fontawesome icons
work, we need to have \texttt{\_included/font-awesome.min.css} in our
project.

There are also additions to \texttt{style.css} to make the inline images
in the custom block (named \texttt{rmdreview}), and the speech bubbles
on the left side of the block:

\begin{verbatim}
.h-icon-chevron-left {
    background: white;
        padding: 3px;
        border: #eee 1px solid;
        color: #666;
      }

.fa-rotate-315 {
    -webkit-transform: rotate(315deg);
    -moz-transform: rotate(315deg);
    -ms-transform: rotate(315deg);
    -o-transform: rotate(315deg);
    transform: rotate(315deg);
}

.rmdreview {
  padding: 1em 1em 1em 5em;
  margin-bottom: 0px;
  background: #f5f5f5 5px center/3em no-repeat;
  position:relative;
}

.rmdreview:before {
    content: "\f0e6";
    font-family: FontAwesome;
    left:10px;
    position:absolute;
    top:10px;
    bottom: 0px;
    font-size: 60px;
 }
\end{verbatim}

\subsection{Call-to-action block}\label{call-to-action-block}

The call-to-action block can appear at the bottom of each page to invite
the reader to submit their email address to receive updates on the book.
This enables us to collect email addresses into a google sheet. The
custom block looks like this:

\BeginKnitrBlock{rmdsale}
Want to know when the book is for sale? Enter your email so I can let
you know. Loading\ldots{}
\EndKnitrBlock{rmdsale}

This custom block is called \texttt{rmdsale}, and is styled with some
css:

\begin{verbatim}
 .rmdsale {
  padding: 1em 1em 1em 4em;
  margin-bottom: 0px;
  background: #f5f5f5 5px center/3em no-repeat;
  position:relative;
}

.rmdsale:before {
    content: "\f07a";
    font-family: FontAwesome;
    left:10px;
    position:absolute;
    top:15px;
    bottom: 0px;
    font-size: 40px;
 }
\end{verbatim}

The text field for readers to enter their email address is created by
code in \texttt{\_includes/email\_submit.html}, which inserts some
elements from a google form into the page.

To make this work, we have to create a new google form (using the `old
google forms', not the current version), and copy over the key and HTML
element details from the default form view. Chrome's element inspector
is very handy for identifying the element details that need to go in
\texttt{email\_submit.html}.

The confirmation page that the reader sees after they submit their email
is not yet elegant, it's just the default google forms page. I'm not
sure what to do about that.

\subsection{Google analytics}\label{google-analytics}

This is already used in many bookdown projects, and is easy to enable.
We go to \url{https://analytics.google.com}, create a new account, and
get the code snippet and paste it into \texttt{google\_analytics.html}
in \texttt{\_includes/}. After that, we add a reference to
\texttt{google\_analytics.html} in \texttt{\_output.yml}, as noted
above. And that's it for setting up the analytics.

\subsection{Not yet implemented}\label{not-yet-implemented}

Matt's ORT contains several other features that are not implemented
here. These include tracking of hypothes.is use, A/B testing, a tool to
translate the book into numerous languages. I'm not sure how to
implement those, and would welcome assistance!

\chapter{Introduction}\label{intro}

\chapter{Literature}\label{literature}

\chapter{Methods}\label{methods}

\chapter{Applications}\label{applications}

\chapter{Final Words}\label{final-words}

\chapter{Open Review}\label{open-review}

\chapter{Placeholder}\label{placeholder}

\bibliography{packages,book}


\end{document}
